This dataset provides anonymised historic trip data from the Divvy bike-sharing service, detailing user travel patterns across various stations and times.
It captures the dynamics of bike usage, offering insights into the frequency and distribution of trips.
Our analysis primarily focuses on the start and end stations' IDs to identify which stations require re-stocking and to infer the most popular destinations.
Additionally, correlating trip start and end times with station usage is essential for a comprehensive understanding.\newline

The dataset includes the following key features:\newline

\subsection{Dataset Features}\label{subsec:dataset-features}
\begin{itemize}
    \item \textbf{star\_ride\_id}: A unique identifier for each trip, ensuring individual trips can be tracked and analyzed discretely.
    \item \textbf{star\_rideable\_type}: The type of bike used for the trip, which can influence usage patterns and availability needs.
    \item \textbf{star\_started\_at}: The timestamp indicating when a trip started, crucial for understanding demand over time.
    \item \textbf{star\_ended\_at}: The timestamp indicating when a trip ended, allowing for the calculation of trip duration and temporal patterns of bike usage.
    \item \textbf{star\_start\_station\_name}: The name of the station where the trip originated, providing a geographic point for demand analysis.
    \item \textbf{star\_start\_station\_id}: A unique identifier for the origin station, which can be used in conjunction with geographic data for mapping and spatial analysis.
    \item \textbf{star\_end\_station\_name}: The name of the station where the trip concluded, indicating the destination demand in the network.
    \item \textbf{star\_end\_station\_id}: A unique identifier for the destination station, useful for spatial analysis and redistribution strategies.
    \item \textbf{star\_start\_lat \& start\_lng}: The latitude and longitude of the start station, giving precise location data for origin points.
    \item \textbf{star\_end\_lat \& end\_lng}: The latitude and longitude of the end station, giving precise location data for destination points.
    \item \textbf{star\_member\_casual}: A categorization of the user as a member or a casual rider, which can influence riding patterns and frequency of use.
\end{itemize}

\subsection{Added Features}\label{subsec:added-features}
The dataset has been enhanced with supplementary features to facilitate analysis and visualization. These characteristics are derived from the existing trip start time and are derived from the initial dataset, enriching the trip records with categorical features.
Zhang et al.~\cite{2016} when describing his approach to features engineering to predict trip destination prediction problem. I approached this matter similarly by mapping it to a binary classification model for each trip origin and pair.\newline
They include:\newline

\begin{itemize}
    \item \textbf{week\_day\_index}: Categorised into Morning, Afternoon, Evening, and Night, to analyse demand fluctuations.
    \item \textbf{day\_period\_index}: Classified into working and non-working days, to observe weekly patterns in bike usage.
\end{itemize}

While the focus is not on text processing, storing text-based features like station names and user types enriches data visualisation, making it more accessible and interpretable.
This approach, although not critical for data model training, enhances the user experience during data exploration. A dataset's value is significantly enhanced by its accessibility.
By detailing and explaining the dataset's features, this report aims to ensure that the data is as useful and informative as possible for the intended analysis.
