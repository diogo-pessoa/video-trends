Bike ride-sharing apps confront the challenge of optimising bike availability across their stations, a dilemma that intensifies near tourist attractions and along commuter routes.
Chiariotti et al.(2018)\cite{s18020512} underscore the significant imbalance within the system,
particularly the rapid depletion of bicycles at residential stations in the morning and congestion at commercial area stations.

This report systematically examines bike share trip records from the publicly sourced Divvy bike share system~\cite{DataSource}.
By deriving and exploring additional features from historical trip data, this study aims to uncover trends in user behaviours regarding bike collection and return.
It employs clustering and classification techniques to build predictive models, addressing key questions to boost operational efficiency: Which stations face the highest demand,
and how do these patterns change throughout the day?
Are there clear trends in user destination preferences, and when do stations experience peak activity?
This investigation also identifies periods requiring urgent bike redistribution to meet high demand and defines `peak hours` based on empirical data analysis.
