The EDA phase is a critical component of the project, aimed at thoroughly examining the cleansed dataset to uncover underlying patterns, trends, and correlations.
This step is foundational for informing subsequent model training and refinement processes.

\subsection{Strategies and Tools}\label{subsec:strategies-and-tools}
\begin{itemize}
    \item \textbf{PySpark for Data Preparation}: Initially, the dataset undergoes preparation and cleaning within the PySpark environment.
    This step is crucial for handling large volumes of data, leveraging PySpark's distributed computing capabilities to efficiently process and ready the data for analysis.
    \item \textbf{Transition to Pandas for Analysis}: Once the data is prepped, it's converted into a Pandas dataframe.
    This conversion is instrumental for the detailed analysis phase, where the rich functionalities of Pandas and its compatibility with visualization libraries come into play.
\end{itemize}

\subsection{Analytical Techniques}\label{subsec:analytical-techniques}
\begin{itemize}
    \item \textbf{Aggregation and Summarization}: Key features of the dataset are aggregated and summarized to distill essential insights.
    This involves counting occurrences, calculating averages, and other statistical measures to better understand the dataset's characteristics.
    \item \textbf{Visualization}: A variety of graphical representations—ranging from basic graphs and tables to more complex heatmaps—are employed to visualize the data.
    These visual aids are invaluable for identifying patterns, detecting anomalies, and understanding the relationships between different dataset features. Graphs and tables offer a straightforward means of presenting quantitative information, making it easier to grasp the distribution and variation within key variables.
    Heatmaps are particularly useful for revealing correlations and patterns across multiple variables, providing a visual summary of complex relationships in a clear and concise manner.
    \item \textbf{Correlation Analysis}: The dataset's features are analyzed for correlations, revealing the strength and direction of relationships between variables.
\end{itemize}





