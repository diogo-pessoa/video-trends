After concluding the exploration and refinement of the data, we proceed with the construction of the predictive models to search for patterns within the bike-sharing dataset.
This section delves into two fundamental modeling methodologies, namely clustering analysis and regression analysis.
Each model has a different purpose.
Clustering helps find patterns in the data, and regression helps predict what will happen in the future.
By evaluating both models, our objective is to identify the most efficient methods for anticipating bike demand across diverse stations and times.

\subsection{Clustering Analysis}\label{subsec:clustering-analysis}


The first approach employs k-means clustering, a method chosen for its efficacy in identifying natural groupings within the dataset based on similarities across several dimensions.
The process involves:

\begin{itemize}
    \item \textbf{Silhouette Score Evaluation}: To determine the optimal number of clusters (k), we utilize the silhouette score, a measure of how similar an object is to its cluster compared to other clusters.
    A higher silhouette score indicates a model with well-separated clusters. The approach used here is based on Shutaywi et al.(2021)~\cite{e23060759}.
    \item \textbf{PCA Visualization}: Post-clustering, we apply Principal Component Analysis (PCA) to reduce the dataset's dimensionality for easier visualization.
    While PCA simplifies the data representation, it can sometimes obscure the original features' interpretability.
    Despite this, insights gained from PCA visualizations are invaluable for understanding the distribution and separation of clusters.
\end{itemize}

\subsection{Classification Analysis}\label{subsec:classification-analysis}
    The second predictive model focuses on regression analysis, specifically employing a multi-classifier approach to predict bike demand.
    This model's objectives include:
\begin{itemize}
    \item \textbf{Classification by Bike type Demand}: The model analysing historical data.
    Then, predicts the demand for bikes based on type at different stations and times, classifying stations according to expected usage levels considering if these are Workday.
\end{itemize}

\subsection{Model Performance Metrics}\label{subsec:performance-review}
    The effectiveness of the regression model is assessed through a variety of performance metrics, including the confusion matrix, precision, and recall.
    These metrics provide a nuanced understanding of the model's accuracy and its ability to generalize across unseen data.
